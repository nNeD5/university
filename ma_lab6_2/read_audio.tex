\documentclass[lab6_2.tex]{subfiles}
\usepackage[total={6in, 10in}]{geometry}

%langue
\usepackage[T1, T2A]{fontenc}
\usepackage[utf8]{inputenc}
\usepackage[english, ukrainian]{babel}


\usepackage[utf8]{inputenc}
\usepackage[T1]{fontenc}
\usepackage{lmodern}
\usepackage{graphicx}
\usepackage{color}
\usepackage{listings}
\usepackage{hyperref}
\usepackage{amsmath}
\usepackage{amsfonts}
\usepackage{epstopdf}
\usepackage{matlab}

\sloppy
\epstopdfsetup{outdir=./}
\graphicspath{ {./read_audio_images/} }

\begin{document}

\begin{par}
\begin{flushleft}
Задаєм константы t та Fs
\end{flushleft}
\end{par}

\begin{matlabcode}
t = 5;
Fs = 8000;
nBits = 16;
nChannels = 1;
deviceId = 1;
\end{matlabcode}

\begin{par}
\begin{flushleft}
Записуєм аудіо
\end{flushleft}
\end{par}

\begin{matlabcode}
disp("Start recording");
\end{matlabcode}
\begin{matlabcode}
audio = audiorecorder(Fs, nBits, nChannels, deviceId);
recordblocking(audio, t)
disp("End recording")
\end{matlabcode}

\begin{par}
\begin{flushleft}
Виводим аудіо
\end{flushleft}
\end{par}

\begin{matlabcode}
play(audio);
\end{matlabcode}

\begin{par}
\begin{flushleft}
Записуєм аудіо до файлу
\end{flushleft}
\end{par}

\begin{matlabcode}
audio_data = getaudiodata(audio);
audiowrite('audio.wav', audio_data, Fs);
save('audio.mat', 'audio_data', 'Fs');
\end{matlabcode}

\end{document}
