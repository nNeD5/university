\documentclass[12pt,a4paper]{article}
\usepackage[utf8]{inputenc}
\usepackage[T1]{fontenc}
\usepackage[ukrainian]{babel}
\usepackage{amsmath}
\usepackage{amsfonts}
\usepackage{amssymb}
\usepackage[dvips]{graphicx}
\usepackage{gensymb}
\usepackage{color,colortbl}
\usepackage{booktabs}
\usepackage[a4paper, total={6in, 8in}]{geometry}
\usepackage{fullpage}
\usepackage{setspace}

%\doublespacing

\setlength{\parindent}{0em}

\author{ФФ-93 Недождій Олексій}

\date{}

\title{Дослід Франка Герца}

\begin{document}
\maketitle
Формули для похибок:

$\delta f_\text{ст} = \dfrac{1}{N}\sqrt{\sum_{i=1}^{N}(f_i - \vec{f})^{2}}$

$\delta f_\text{сис} = \sqrt{\sum(\frac{\partial f}{\partial x_i} \cdot \delta x_i)}$

$\delta f_\text{заг} = \sqrt{\delta^{2} f_\text{сси} + \delta^{2} f_\text{ст}}$

\section*{Данні Експерименту}

\begin{table}[!h]
    \centering
    \caption{\text{Експериментальні данні з осцилографа}}
    \begin{tabular}{ccc}
        \hline
        $\overline{U}$, В & $\delta U$, В \\
        \hline
        2.125 & 0.101 & перeгин\\
        4.300 & 0.050 & махимум\\
        8.100 & 0.158 & перегин\\
        10.425 & 0.062 & махимум\\
        12.425 & 0.101 & перегин\\
        15.525 & 0.135 & махимум\\
        18.925 & 0.135 & перегин\\
        22.075 & 0.135 & махимум\\
        25.250 & 0.090 & перегин\\
        29.700 & 0.112 & махимум\\
        31.850 & 0.071 & перегин\\
        34.375 &0.090 & махимум\\
    \end{tabular}
\end{table}

\[
    \delta U = \sqrt{0.05^{2} + (\frac{1}{4}\sum_{i=1}^4 (U_i - \overline{U})^{2})^{2}}
.\]

\begin{table}[!h]
    \centering
    \caption{\text{Експериментальні данні з аналізатора в кГц}}
    \begin{tabular}{cc}
        \hline
        $\overline{max}_{left}$ & $\overline{max}_{right}$ \\
        \hline
        0.0255 & 0.0143 \\
        0.0425 & 0.0348 \\
        0.0635 & 0.0563 \\
        0.0848 & 0.0783 \\
        0.1050 & 0.0973 \\
        0.1440 & 0.1398 \\
        0.1850 & 0.1830  \\
        \hline
    \end{tabular}
 \end{table}

\newpage
\begin{table}[!h]
    \centering
    \caption{\text{Середнє між $max_{left}$ та $max_{right}$ в кГц}}
    \begin{tabular}{cc}
        \hline
        $\nu $ & $\delta \nu$ \\
        \hline
        19.875 & 6.920\\
        38.625 & 5.592\\
        59.875 & 5.421\\
        81.500 & 5.257\\
        101.125 & 5.592\\
        141.875 & 4.557\\
        184.000 & 4.153\\
        \hline
    \end{tabular}
\end{table}

В похибку був врахований ефект дрейфу нуля.
В формулу для похибки додали центральне значення частоти.
\[
    \delta \nu = \sqrt{0.0005^2 + 0.004^{2} +
    \frac{1}{8}\sum_{i=1}^8 (\nu_i - \overline{\nu})^{2}}
.\]

\section*{Осцилограф}
\[
    \Delta E_1 = e ({U_max2} - U_{max1}) = 11.2
.\]



\section*{Визначперегинення енергії переходу}

З даних, отриманих на аналізаторі(середні значення):

Щоб визначити енергію з досліду з аналізатором, скористаємось співвідношенням:

\begin{equation}
	E_x = \dfrac{E_0 \nu_0}{\nu_x}
\end{equation}

де $E_0 = 42$ еВ -- енергія живлення, їй відповідає частота $\nu_0 = 19$ Гц.

Інструментальна похибка $\sigma_\nu = 4$ Гц;

Похибка для енергії $\sigma_{E_1} = \delta_E \cdot E = \dfrac{E_0 \nu_0}{\nu_x} \dfrac{\sigma_\nu}{nu}, \; \sigma_{E_2} = \dfrac{E_0}{\nu_x} \sigma_\nu,$

$\sigma_E = \sqrt{(\sigma_{E_1})^2 + (\sigma_{E_2})^2}$.

Отже, моємо такі значення:

\begin{table}[h]
	\centering
	\begin{tabular}{llll}
		\hline
		$\nu$, Гц & $\sigma_\nu$, Гц & $E$, еВ & $\sigma_E$, еВ\\

		\hline
	\end{tabular}
\end{table}

\begin{center}
    \begin{tabular}{l|c}
        \hline
        $E$                         & $\nu$ \\
        \hline
         Перша енергiя збудження    & 31.2941 \\
         \hline
         Друга енергiя збудження    & 18.7765 \\
         \hline
         Третя енергiя збудження    & 12.5669 \\
         \hline
         Четверта енергiя збудження &  9.4104 \\
         \hline
         П'ята енергiя збудження    &  7.6    \\
         \hline
         Шоста енергiя збудження    &  5.5417 \\
         \hline
         Сьома енергiя збудження    &  4.3135 \\
        \hline
    \end{tabular}
\end{center}

\end{document}
